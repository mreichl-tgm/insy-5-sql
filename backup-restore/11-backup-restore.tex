\documentclass{school}
\title{Fragen zu „Backup \& Restore“}
\subject{Informationssysteme}
\author{Markus Reichl}

\begin{document}
\vskip 1em
Die folgenden Aufgabenstellungen sollen jeweils mit MySQL und PostgreSQL umgesetzt werden. Verwende dazu die Standard-Backup-Tools dieser DBMS (MySQL: mysqldump, mysqlhotcopy, ibbackup bzw. PostgreSQL: pg\_dump).
Die Durchführung der Schritte soll mittels Screenshots dokumentiert werden und jeweils durch eine kurze Zusammenfassung der Ergebnisse bzw.Erkenntnisse erläutert werden. Die verwendeten Quellen sollen dokumentiert werden. 
Die Abgabe soll als .pdf-Dokument erfolgen.

\newpage
\section{Wie können eine/mehrere/alle Datenbanken des Servers in einer gemeinsamen Datei gespeichert werden?}
\verb|$ mysqldump --opt -u [uname] -p[pass] [db] > [file].sql|\\
\verb|$ pg_dump [connection-option...] [option...] [dbname] > [file].sql|
\subsection{Eine Datenbank}
\verb|$ mysqldump -u root -p [db] > [file].sql|\\
\verb|$ pg_dump -U root -W [dbname] > [file].sql|
\subsection{Mehrere Datenbanken}
\verb|$ mysqldump -u root -p --databases [db1] [db2] > [file].sql|
\subsection{All Datenbanken}
\verb|$ mysqldump -u root -p --all-databases > [file].sql|\\
\verb|$ pg_dumpall -U root -W [dbname] > [file].sql|

\section{Wie kann definiert werden, dass zusätzlich zu den Daten auch die Tabellenstrukturen/Trigger/Stored Routines gespeichert werden?}
\subsection{MySQL}
\paragraph{Tabellenstrukturen: }\verb|--create-options|
\paragraph{Trigger: } Deaktiviert durch \verb|--skip-triggers|
\paragraph{Stored Routines: }\verb|--routines -R|

\subsection{Postgres}
Werden standardmäßig gespeichert. Trigger können über \verb|--disable-triggers| deaktiviert werden.

\section{Wie können die IF EXISTS bzw. DROP-Klauseln in den Sicherungen eingefügt/unterdrückt werden?}
\subsection{MySQL}
\verb|--add-drop-database|
\verb|--add-drop-table|
\verb|--add-drop-trigger|

\subsection{Postgres}
\verb|-c --clean|

\newpage
\section{Was sind die Vor- bzw. Nachteile der beiden Verfahren „Physisches Backup“/„Logisches Backup“?}
\subsection{Physisch}
+ Rascher einsetzbar (kein Parser)\\
- Nicht lesbar
\subsection{Logisch}
- Wesentlich mehr Speicherplatz

\section{Wie kann ein „Hot Backup“durchgeführt werden?}
\subsection{MySQL}
https://www.digitalocean.com/community/tutorials/how-to-create-hot-backups-of-mysql-databases-with-percona-xtrabackup-on-ubuntu-14-04

\subsection{Postgres}
http://www.anchor.com.au/hosting/dedicated/Postgres-backup-improvement-with-hot-backup

\section{Wie kann ein „Remote Backup“(z.B. auf einem externen, gemieteten Server) angestoßen werden?}
RClone

\section{Wie könnten die Backupvarianten aus Punkt 1.) automatisiert werden (Uhrzeit als Trigger)?}
In regelmäßigen Intervallen / Copy on Write

\section{Geben Sie entsprechend für ihr Betriebssystem (Windows, Linux, Mac, ...) Möglichkeiten an.}
Systemd, Cronjobs (Cronie)

\section{Wie könnte der Zeitpunkt der Sicherung in den Dateinamen des Dumps automatisiert eingefügt werden (z.B. DBNAME\_20100413\_0952.sql)?}
Systemvariablen setzen und als Dateiname verwenden: Basic Batch Scripting

\end{document}